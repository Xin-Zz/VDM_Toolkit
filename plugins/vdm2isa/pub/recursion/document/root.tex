\documentclass[runningheads,a4paper]{llncs}
%\documentclass[11pt,a4paper]{article}
\usepackage{isabelle,isabellesym}

% further packages required for unusual symbols (see also
% isabellesym.sty), use only when needed

%\usepackage{amssymb}
  %for \<leadsto>, \<box>, \<diamond>, \<sqsupset>, \<mho>, \<Join>,
  %\<lhd>, \<lesssim>, \<greatersim>, \<lessapprox>, \<greaterapprox>,
  %\<triangleq>, \<yen>, \<lozenge>

%\usepackage{eurosym}
  %for \<euro>

%\usepackage[only,bigsqcap]{stmaryrd}
  %for \<Sqinter>

%\usepackage{eufrak}
  %for \<AA> ... \<ZZ>, \<aa> ... \<zz> (also included in amssymb)

%\usepackage{textcomp}
  %for \<onequarter>, \<onehalf>, \<threequarters>, \<degree>, \<cent>,
  %\<currency>

% urls in roman style, theory text in math-similar italics
%\urlstyle{rm}
\isabellestyle{it}

% this should be the last package used
\usepackage{pdfsetup}

% for uniform font size
%\renewcommand{\isastyle}{\isastyleminor}
\usepackage[color]{vdmlisting}

\begin{document}

%\title{A translation strategy for VDM recursive functions in Isabelle/HOL}
\title{VDM recursive functions in Isabelle/HOL}
%\author{Leo Freitas, Peter Gorm Larsen}
%Multiple institutes are typeset as follows:
\author{Leo Freitas\inst{1} \and Peter Gorm Larsen\inst{2}}

%If there are too many authors, use \authorrunning
% \authorrunning{First Author et al.}
\authorrunning{ }

\institute{
School of Computing, Newcastle University, \\
\email{leo.freitas@newcastle.ac.uk}
\and 
DIGIT, Aarhus University, Department of Engineering, \\
%Finlandsgade 22, 8200 Aarhus N, Denmark\\
\email{pgl@ece.au.dk}
}
			
\maketitle
\setcounter{footnote}{0} 
\begin{abstract}
For recursive functions the general principles of induction needs to be applied. Instead of verifying directly in using the Vienna Development Method Specification Language (VDM-SL), we suggest a translation to Isabelle/HOL. In this paper, the challenges of such a translation for recursive functions are presented. This is an extension of an existing translation and a VDM mathematical toolbox in Isabelle/HOL enabling support for recursive functions.
\end{abstract}

\keywords{VDM, Isabelle/HOL, Translation, Recursion, VSCode}

%\tableofcontents

% sane default for proof documents
\parindent 0pt\parskip 0.5ex

% generated text of all theories
\input{session}

% optional bibliography
\bibliographystyle{splncs03}
 \bibliography{root}

\end{document}

%%% Local Variables:
%%% mode: latex
%%% TeX-master: t
%%% End:
